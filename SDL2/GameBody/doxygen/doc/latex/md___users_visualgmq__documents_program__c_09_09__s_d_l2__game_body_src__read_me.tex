Game\+Body将\+S\+D\+L2的基本流程封装在\+Game\+Body类中,方便以后写测试文件的时候可以快速的测试。 \subsection*{Game\+Body使用方法:}

原生的\+Game\+Body类只是创造一个窗口,不会有任何的其他动作。你可以创建一个\+Game\+Body对象来显示这个窗口。如果想要添加自己的东西,需要继承\+Game\+Body类,重写update()函数以及clean()函数。update用于在游戏中执行的更新,clean用于在游戏结束后的清理工作。 Game\+Body类里除了is\+Quit()和构造函数之外,所有的函数都是virtual函数,意味着你可以重写来实现自己的功能。一般都是重写update()函数来实现功能。 \subsection*{Game\+Body对象的使用流程:}


\begin{DoxyEnumerate}
\item 首先声明一个\+Game\+Body或者你自己继承的类的对象。
\item 使用\+R\+U\+N\+\_\+\+A\+P\+P()宏来让你的类运行起来。
\end{DoxyEnumerate}

就可以完成游戏的流程。

更新: {\bfseries{2018.\+12.\+6}}\+: 完成了\+Game\+Body的类定义。

{\bfseries{2018.\+12.\+7}}\+: 添加了\+::ifdef \+\_\+\+X\+C\+O\+D\+E\+\_\+\+P\+R\+O\+J\+E\+C\+T\+\_\+来使这个头文件在\+X\+C\+O\+D\+E和g++里面都可以编译(因为这两个的头文件导入的路径不一样)。但是不可以简单的直接在源文件里面加上\#define \+\_\+\+X\+C\+O\+D\+E\+\_\+\+P\+R\+O\+J\+E\+C\+T\+\_\+,而是需要在\+Game\+Body.\+cpp和\+Game\+Body.\+hpp的开始部分都加上 \#ifndef \+\_\+\+X\+C\+O\+D\+E\+\_\+\+P\+R\+O\+J\+E\+C\+T\+\_\+ \#define \+\_\+\+X\+C\+O\+D\+E\+\_\+\+P\+R\+O\+J\+E\+C\+T\+\_\+ \#endif

{\bfseries{2019.\+2.\+16}} 将main函数封装在\+R\+U\+N\+\_\+\+A\+P\+P()宏里面,方便整体结构,不需要写main函数,也不需要记住类的成员函数的调用顺序了。 